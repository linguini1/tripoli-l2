\section{Design Overview}

This rocket will be a dual break, dual deploy rocket. It is technically only
necessary to have a single-break, single-deploy system, but this design choice
will teach me more about what's required for an L3 certification as well as
reduce any ability for the rocket to drift after apogee.

Since the rocket is dual break, the design for the avionics bay is going to be a
switch-band. This is a common choice for dual-deploy rocket and also looks very
simple to integrate (to me). Since the entire switch-band assembly is removable,
it will be easy to fiddle with when installing the avionics and can be replaced
entirely if I do something silly like drill the arming holes terribly wrong or
discover I need to re-arrange the entire avionics section. The primary
inspiration for the design the switchband avionics section is the design from
Joe Barnard's \textit{Send It!} rocket. \cite[\nopp 4:14]{bps-sendit} I like
the design choice of using Wago lever connectors for deployment charges to avoid
a screwdriver \cite[\nopp 5:47]{bps-sendit}, but I will need to come up with a
better solution than just epoxying the connectors to the bulkhead in case one
needs to be replaced. The bulkheads will be pressure sealing in order to prevent
the deployment charges from affecting the barometric altimeters inside the
avionics bay, and also to make sure that the correct volume of the rocket is
pressurized to deploy the chutes. Most of the configurations I have seen use a
small lip on both bulkheads so that they rest on the switchband coupler tube. I
have been told that using a o-ring to seal the pressure is overkill.

\begin{figure}[H]
	\centering
	\includegraphics[width=0.6\linewidth]{assets/bps-bulkhead-lip.png}
	\caption{Aluminum bulkhead for switchband with a visible lip for the coupler
    tube \cite[\nopp 6:17]{bps-sendit}}
\end{figure}

For deployment electronics, I will not fly a redundant configuration if I can
help it (the Tripoli L2 rules do not require it \cite{tripoli-l2}) in order to
save costs, but I would like the avionics bay design to be capable of supporting
a redundant configuration if necessary. This should mostly be an issue of wiring
and extra space in the avionics bay, so it is unlikely to be a large
undertaking.
