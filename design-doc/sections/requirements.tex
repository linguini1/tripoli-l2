\section{Design Requirements}

The primary requirements for the rocket are derived from the Tripoli L2
certification requirements in \cite{tripoli-l2}, the L2 experience flight
requirements as per \cite[\nopp 4]{tripoli-l3} and the Tripoli Unified Safety
Code. \cite{tripoli-safety} These requirements are critical to meet in order to
achieve the goal of this rocket, which is to obtain my L2 certification and
complete the experience requirements towards an L3.

The secondary requirements for the rocket are derived from the desire to lower
the risk of failing the certification process and to reduce the amount of effort
required for assembly, pre-flight preparation and post-flight recovery.

The tertiary requirements for the rocket are derived from the desire to learn
some new skills and have a somewhat cool certification rocket to be proud of.
These requirements can be given up, just at my own disappointment.

\subsection{Primary Requirements}

\begin{enumerate}[label={R 1.\arabic*}]
	\item The rocket must be of conventional design (stabilized by fins and
	      recovered under parachutes) \cite[Airframe]{tripoli-l2}
	\item The rocket must be not be entirely 3D printed \cite[Airframe]{tripoli-l2}
	\item The rocket must be recovered by parachute \cite[Recovery]{tripoli-l2}
	\item The rocket must have a descent rate less than 11 meters per second
	      \cite[\nopp 11-1]{tripoli-safety}
	\item The rocket's recovery deployment must be electronically controlled
	      \cite[4]{tripoli-l3}
	\item All components of the rocket must be connected and recovered in one
	      assembly \cite[Non-Certification]{tripoli-l2}
	\item The rocket must be powered by a single J, K or L motor \cite[Motor]{tripoli-l2}
	\item The rocket must have a thrust to weight ratio of 5:1 or more \cite[\nopp 7-1.4]{tripoli-safety}
	\item The rocket must have a stable ascent \cite[Certification Flight]{tripoli-l2}
	\item The rocket must have a marked center of pressure \cite[7-1.2]{tripoli-safety}
	\item The rocket must have a marked center of gravity \cite[7-1.2]{tripoli-safety}
\end{enumerate}

\subsection{Secondary Requirements}

\begin{enumerate}[label={R 2.\arabic*}]
    \item The rocket must fly a 38mm motor
	\item The rocket must be capable of recovery deployment both via motor
	      ejection charge and electronic control, under the assumption that both
	      fire during flight.
	\item Continuity of deployment charge channels must be verifiable at the
	      time of arming on the pad.
	\item The rocket must have a GPS tracker so that its final landing location
	      can be obtained while tracking its flight from the vertical launch area.
	\item The deployment altimeter must be equipped with a barometer for
	      higher accuracy altitude estimation.
	\item The deployment system must be capable of firing deployment charges
	      both at a specific altitude and after a fixed time since lift-off in
	      the case of failure to detect altitude.
	\item The GPS tracker frequency must be capable of being easily changed on
	      the pad.
	\item The deployment altimeter must log timestamped altitude measurements
	      during flight.
	\item The deployment altimeter must log timestamped deployment event
	      signals.
	\item All locations for connecting shock-cord which need to be accessed
	      during pre-flight preparation must be easily accessible by hand.
	\item Rail buttons must be capable of being installed using the provided
	      flanges.
	\item One rail button must be placed as low as possible on the rocket
	      airframe, and the other must be placed at the expected center of
	      gravity.
	\item The GPS tracker must have a transmission range that exceeds the
	      maximum expected drift of the rocket.
	\item All 3D printed parts and assembly tools must be printable on a
	      $220\times220\times250$ mm print bed.
	\item The deployment electronics bay must have sufficient venting holes
	      for access to the atmosphere during flight.
	\item All separate volumes of the rocket must have sufficient venting holes
	      to prevent unintended separation.
	\item The rocket avionics bay must be capable of supporting a redundant
	      deployment configuration.
\end{enumerate}

\subsection{Tertiary Requirements}

\begin{enumerate}[label={R 3.\arabic*}]
	\item At least one part of the airframe should be made of a composite material
	      (either fiberglass or carbon fiber is acceptable)
	\item No fastener should be a Phillips head
	\item The rocket should be easy to manufacture
    \item The rocket might have a boat-tail
\end{enumerate}
